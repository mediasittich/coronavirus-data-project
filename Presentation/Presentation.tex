\documentclass{beamer}

\setbeamertemplate{navigation symbols}{}

\usepackage{graphicx}
\usepackage{amsmath, amssymb, amsfonts, amsthm}
\usepackage{tabularx}

\usetheme{CambridgeUS}


%-----------------------------------------------------------------------------------------------------------------
% New Environment for a plain box
\newenvironment{boxeded}
    {\begin{center}
    \begin{tabular}{|p{\textwidth}|}
    \hline\\
    }
    { 
    \\\\\hline
    \end{tabular} 
    \end{center}
    }
%-----------------------------------------------------------------------------------------------------------------

% Die Präsentation dient dem Praxisprojekt: COVID-19 und wurde von Regina Galambos und Lorenz Mihatsch erstellt.

% Folgende Inhalte sollen besprochen werden: 
% 1: Datenstruktur und -erhebung erläutern:
% 1.1) Probleme der Erhebung: Recoding policy, testing policy.
% 2: Fallzahlen Weltweit als Timescale und als Weltkarte bezogen auf 100k Einwohner
% 3: Wachstumsraten der Fallzahlen
% 4: Ländervergleich. Zentrieren um Tag der Abriegelung irgendeiner Art.






%-----------------------------------------------------------------------------------------------------------------
% Presentation information: title, author.....

\title[Praxisprojekt: COVID-19]{Analyse der COVID-19 Fallzahlen}
\subtitle{Praxisprojekt}
\author[R.Galambos, L.Mihatsch]{Regina Galambos, Lorenz Mihatsch\\
	\includegraphics[width=0.22\textwidth]{LMU.pdf}\\
	{\small Projektpartner: Andr\'{e} Klima}}
%-----------------------------------------------------------------------------------------------------------------


%-----------------------------------------------------------------------------------------------------------------
% Begin of presentation.

%Title page
\begin{document}
\begin{frame}
	\titlepage
\end{frame}

% Table of contents
\begin{frame}
   \frametitle{Inhaltsangabe}
   \tableofcontents
 \end{frame}
 
 
 % Einführung
 %-----------------------------------------------------------------------------------------------------------------
 \section{Einführung}
 \begin{frame}
 	\frametitle{COVID-19 Pandemie}
	Kommentar: Pandemie erklären. Datenerhebung der John Hopkins Universität.
	Weg App erwähnen mit Interaktiven Graphiken und dem Link.
 \end{frame}
 \section{Daten}
  
 \begin{frame}
 	\frametitle{Daten}
	Erklärungen zum Datensatz und Abruf der Daten über RamiKrispin
	Weitere Teildatensätze: Kontinente, Population und Länderfläche.
	Klar machen, dass es sich um Reported Cases handelt.
 \end{frame}
 
 \section{Weltweit}
 \begin{frame}
 	\frametitle{Kumulative Daten weltweit}
	Kommentar: Cumulative Daten der Welt als Timeline. Graphik mit Cases, Death und Recovered.
	Aus Zeitlichen Gründen wird "Recovered" aus der Präsentation weggelassen.
	Plot der Kumulativen Fallzahlen ab 100 Fälle mit den Linien der Verdoppelungszeiten?
	Umgang mit Diamond Pincess und MS Zaandam erklären: Rauslassen
 \end{frame}
 
\section{Wachstumsraten}
\begin{frame}
\frametitle{Wachstumsraten}
	Kommentar: Berechnung der Wachstumsraten erklären, rolling geometric mean erklären,
	Verdopplungzeit Formel erklären erklären
	Plot aller Länder mit world daten
\end{frame}

\begin{frame}
\frametitle{Wachstumsraten}
	\begin{figure}
		\centering
		\includegraphics[width = 270pt]{GF_confirmed}
	\end{figure}
\end{frame}

\begin{frame}
\frametitle{Wachstumsraten}
	\begin{figure}
		\centering
		\includegraphics[width = 270pt]{GF_deaths}
	\end{figure}
\end{frame}

 \section{Ländervergleich}
\begin{frame}
\frametitle{Infektionsmaßnahmen}
	Kommentar: Beispiel Plot von South Korea um Problematik der Zentrierung zu erläutern. Wachstumsraten bzw. Verdoppelungszeit zentriert um die Einführung der Maßnahmen.
\end{frame}
 
\end{document}